\documentclass[a4paper, oneside, english,reqno]{amsart}


\usepackage[utf8]{inputenx}
\usepackage{amsthm, amsmath, amssymb, amsfonts, mathrsfs, mathtools, stmaryrd}
\usepackage{babel, textcomp, url, enumerate, latexsym, graphicx, varioref, hyperref, multirow, layout, siunitx, booktabs}
\usepackage{pdflscape}
\usepackage{verbatim}
\usepackage{geometry}

\usepackage{tikz, babel, rotating, tikz-cd, pigpen}
\usetikzlibrary{matrix, arrows, decorations.pathmorphing}
%\usepackage{ntheorem}
\usepackage{cleveref}
\RequirePackage[color       = blue!20,
                bordercolor = black,
                textsize    = tiny,
                figwidth    = 0.99\linewidth]{todonotes}

%\presetkeys{shadows}
%\usetikzlibrary{matrix, arrows}
\usepackage[all]{xy}
\usepackage{microtype}
%\usepackage{enumitem}
 %\usepackage{manfnt}

\usepackage[backend = biber, style = alphabetic, url = false, isbn = false, doi = false]{biblatex}
\addbibresource{ref.bib}

\DeclarePairedDelimiter{\p}{\lparen}{\rparen}
\DeclarePairedDelimiter{\set}{\{}{\}}
\DeclarePairedDelimiter{\abs}{|}{|}
\DeclarePairedDelimiter{\norm}{\|}{\|}
\DeclarePairedDelimiter{\bracket}{[}{]}
\DeclarePairedDelimiter{\bbracket}{\llbracket}{\rrbracket}
\DeclarePairedDelimiter{\ip}{\langle}{\rangle}
\DeclarePairedDelimiter{\pfister}{\langle\!\langle}{\rangle\!\rangle}

%\counterwithout{section}{chapter}

% Theorems
\theoremstyle{plain}
\newtheorem{theorem}{Theorem}[section]
\newtheorem{lemma}[theorem]{Lemma}
\newtheorem{scholium}[theorem]{Scholium}
\newtheorem{prop}[theorem]{Proposition}
\newtheorem{conjecture}[theorem]{Conjecture}
\newtheorem{corollary}[theorem]{Corollary}
\newtheorem{problem}[theorem]{Problem}
\newtheorem{statement}[theorem]{Statement}

%\newtheoremstyle{named}{}{}{\itshape}{}{\bfseries}{.}{.5em}{#1 \thmnote{#3}}
%\theoremstyle{named}
%\newtheorem*{namedtheorem}{Theorem}

\theoremstyle{definition}
\newtheorem{definition}[theorem]{Definition}
\newtheorem{assumption}[theorem]{Assumption}
\newtheorem{hypothesis}[theorem]{Hypothesis}
\newtheorem{example}[theorem]{Example}
\newtheorem{exercise}[theorem]{Example}
\newtheorem{question}[theorem]{Question}
\newtheorem{construction}[theorem]{Construction}
\newtheorem{idea}[theorem]{Idea}

\theoremstyle{remark}
\newtheorem{remark}[theorem]{Remark}
\newtheorem{notation}[theorem]{Notation}
\newtheorem{remarks}[theorem]{Remarks}

%\newcommand{\pushout}{\arrow[dr,phantom,\text{\pigpenfont R}]}
%\newcommand{\pushout}{\arrow[dr,phantom, ``\ulcorner'', very near start]}
\newcommand{\latHom}[3]{\operatorname{Hom}_{#1}(#2,#3)}
\newcommand{\gr}[1]{#1^{\textbf{gr}}}
\newcommand{\quot}[2]{#1/#2 #1}
\newcommand{\wt}{\widetilde}
\newcommand{\wh}{\widehat}
\newcommand{\ol}{\overline}
\newcommand{\ul}{\underline}
\newcommand{\defeq}{\vcentcolon=}
\newcommand{\eqdef}{=\vcentcolon}
\newcommand{\ee}{é}
\newcommand{\nr}{\mathrm{nr}}
\newcommand{\nc}{\mathrm{nc}}
\newcommand{\reg}{\mathrm{reg}}
\newcommand{\et}{\textup{\ee{}t}}
%\newcommand{\MW}{\textup{MW}}
%\newcommand{\Milnor}{\textup{M}}
\newcommand{\iso}{\mathrm{iso}}
\newcommand{\Nis}{\mathrm{Nis}}
\newcommand{\Zar}{\mathrm{Zar}}
\newcommand{\op}{\mathrm{op}}
\newcommand{\re}{\mathrm{Re}}
\newcommand{\ordinary}{\mathrm{ord}}
\newcommand{\DD}{\Delta}
\newcommand{\bb}{\bullet}
\newcommand{\DQ}{\mathbf{DQ}}
\newcommand{\ACor}{A\mkern-3mu\Cor}

\newcommand{\rmc}{\mathrm{c}}
\newcommand{\sspt}{\mathbf{1}}
\newcommand{\A}{\mathbf{A}}
\newcommand{\B}{\mathbf{B}}
\newcommand{\C}{\mathbf{C}}
\newcommand{\D}{\mathbf{D}}
\newcommand{\E}{\mathbf{E}}
\newcommand{\F}{\mathbf{F}}
\newcommand{\G}{\mathbf{G}}
\newcommand{\HH}{\mathbf{H}}
\newcommand{\I}{\mathbf{I}}
\newcommand{\J}{\mathbf{J}}
\newcommand{\K}{\mathbf{K}}
\newcommand{\LL}{\mathbf{L}}
\newcommand{\M}{\mathbf{M}}
\newcommand{\N}{\mathbf{N}}
\newcommand{\OO}{\mathbf{O}}
\newcommand{\PP}{\mathbf{P}}
\newcommand{\Q}{\mathbf{Q}}
\newcommand{\R}{\mathbf{R}}
\newcommand{\bfS}{\mathbf{S}}
\newcommand{\T}{\mathbf{T}}
\newcommand{\U}{\mathbf{U}}
\newcommand{\V}{\mathbf{V}}
\newcommand{\W}{\mathbf{W}}
\newcommand{\X}{\mathbf{X}}
\newcommand{\Y}{\mathbf{Y}}
\newcommand{\Z}{\mathbf{Z}}

\newcommand{\Zp}{\mathbf{Z}_p}
%\newcommand{\R}{\mathbf{R}}
%\newcommand{\HH}{\mathbf{H}}
%\newcommand{\G}{\mathbf{G}}
%\newcommand{\SS}{\mathbf{S}}

\newcommand{\bfa}{\mathbf{a}}
\newcommand{\bfb}{\mathbf{b}}
\newcommand{\bfc}{\mathbf{c}}
\newcommand{\bfd}{\mathbf{d}}
\newcommand{\bfe}{\mathbf{e}}
\newcommand{\bff}{\mathbf{f}}
\newcommand{\bfg}{\mathbf{g}}
\newcommand{\bfh}{\mathbf{h}}
\newcommand{\bfi}{\mathbf{i}}
\newcommand{\bfj}{\mathbf{j}}
\newcommand{\bfk}{\mathbf{k}}
\newcommand{\bfl}{\mathbf{l}}
\newcommand{\bfm}{\mathbf{m}}
\newcommand{\bfn}{\mathbf{n}}
\newcommand{\bfo}{\mathbf{o}}
\newcommand{\bfp}{\mathbf{p}}
\newcommand{\bfq}{\mathbf{q}}
\newcommand{\bfr}{\mathbf{r}}
\newcommand{\bfs}{\mathbf{s}}
\newcommand{\bft}{\mathbf{t}}
\newcommand{\bfu}{\mathbf{u}}
\newcommand{\bfv}{\mathbf{v}}
\newcommand{\bfw}{\mathbf{w}}
\newcommand{\bfx}{\mathbf{x}}
\newcommand{\bfy}{\mathbf{y}}
\newcommand{\bfz}{\mathbf{z}}

\newcommand{\fraka}{\mathfrak{a}}
\newcommand{\frakA}{\mathfrak{A}}
\newcommand{\frakb}{\mathfrak{b}}
\newcommand{\frakB}{\mathfrak{B}}
\newcommand{\frakc}{\mathfrak{c}}
\newcommand{\frakC}{\mathfrak{C}}
\newcommand{\frakd}{\mathfrak{d}}
\newcommand{\frakD}{\mathfrak{D}}
\newcommand{\frakE}{\mathfrak{E}}
\newcommand{\frakF}{\mathfrak{F}}
\newcommand{\frakG}{\mathfrak{G}}
\newcommand{\frakH}{\mathfrak{H}}
\newcommand{\frakI}{\mathfrak{I}}
\newcommand{\frakJ}{\mathfrak{J}}
\newcommand{\frakK}{\mathfrak{K}}
\newcommand{\frakL}{\mathfrak{L}}
\newcommand{\frakm}{\mathfrak{m}}
\newcommand{\frakM}{\mathfrak{M}}
\newcommand{\frakn}{\mathfrak{n}}
\newcommand{\frakN}{\mathfrak{N}}
\newcommand{\frako}{\mathfrak{o}}
\newcommand{\frakp}{\mathfrak{p}}
\newcommand{\frakP}{\mathfrak{P}}
\newcommand{\frakq}{\mathfrak{q}}
\newcommand{\frakQ}{\mathfrak{Q}}
\newcommand{\frakr}{\mathfrak{r}}
\newcommand{\frakR}{\mathfrak{R}}
\newcommand{\fraks}{\mathfrak{s}}
\newcommand{\frakS}{\mathfrak{S}}
\newcommand{\frakt}{\mathfrak{t}}
\newcommand{\frakT}{\mathfrak{T}}
\newcommand{\fraku}{\mathfrak{u}}
\newcommand{\frakU}{\mathfrak{U}}
\newcommand{\frakv}{\mathfrak{v}}
\newcommand{\frakV}{\mathfrak{V}}
\newcommand{\frakw}{\mathfrak{w}}
\newcommand{\frakW}{\mathfrak{W}}
\newcommand{\frakx}{\mathfrak{x}}
\newcommand{\frakX}{\mathfrak{X}}
\newcommand{\fraky}{\mathfrak{y}}
\newcommand{\frakY}{\mathfrak{Y}}
\newcommand{\frakz}{\mathfrak{z}}
\newcommand{\frakZ}{\mathfrak{Z}}

\newcommand{\sfa}{\mathsf{a}}
\newcommand{\sfA}{\mathsf{A}}
\newcommand{\sfb}{\mathsf{b}}
\newcommand{\sfB}{\mathsf{B}}
\newcommand{\sfc}{\mathsf{c}}
\newcommand{\sfC}{\mathsf{C}}
\newcommand{\sfd}{\mathsf{d}}
\newcommand{\sfD}{\mathsf{D}}
\newcommand{\sfE}{\mathsf{E}}
\newcommand{\sff}{\mathsf{f}}
\newcommand{\sfF}{\mathsf{F}}
\newcommand{\sfG}{\mathsf{G}}
\newcommand{\sfH}{\mathsf{H}}
\newcommand{\sfI}{\mathsf{I}}
\newcommand{\sfJ}{\mathsf{J}}
\newcommand{\sfK}{\mathsf{K}}
\newcommand{\sfL}{\mathsf{L}}
\newcommand{\sfm}{\mathsf{m}}
\newcommand{\sfM}{\mathsf{M}}
\newcommand{\sfn}{\mathsf{n}}
\newcommand{\sfN}{\mathsf{N}}
\newcommand{\sfo}{\mathsf{o}}
\newcommand{\sfp}{\mathsf{p}}
\newcommand{\sfP}{\mathsf{P}}
\newcommand{\sfq}{\mathsf{q}}
\newcommand{\sfQ}{\mathsf{Q}}
\newcommand{\sfr}{\mathsf{r}}
\newcommand{\sfR}{\mathsf{R}}
\newcommand{\sfs}{\mathsf{s}}
\newcommand{\sfS}{\mathsf{S}}
\newcommand{\sft}{\mathsf{t}}
\newcommand{\sfT}{\mathsf{T}}
\newcommand{\sfu}{\mathsf{u}}
\newcommand{\sfU}{\mathsf{U}}
\newcommand{\sfv}{\mathsf{v}}
\newcommand{\sfV}{\mathsf{V}}
\newcommand{\sfw}{\mathsf{w}}
\newcommand{\sfW}{\mathsf{W}}
\newcommand{\sfx}{\mathsf{x}}
\newcommand{\sfX}{\mathsf{X}}
\newcommand{\sfy}{\mathsf{y}}
\newcommand{\sfY}{\mathsf{Y}}
\newcommand{\sfz}{\mathsf{z}}
\newcommand{\sfZ}{\mathsf{Z}}

\newcommand{\scrA}{\mathscr{A}}
\newcommand{\scrB}{\mathscr{B}}
\newcommand{\scrC}{\mathscr{C}}
\newcommand{\scrD}{\mathscr{D}}
\newcommand{\scrE}{\mathscr{E}}
\newcommand{\scrF}{\mathscr{F}}
\newcommand{\scrG}{\mathscr{G}}
\newcommand{\scrH}{\mathscr{H}}
\newcommand{\scrI}{\mathscr{I}}
\newcommand{\scrJ}{\mathscr{J}}
\newcommand{\scrK}{\mathscr{K}}
\newcommand{\scrL}{\mathscr{L}}
\newcommand{\scrM}{\mathscr{M}}
\newcommand{\scrN}{\mathscr{N}}
\newcommand{\scrO}{\mathscr{O}}
\newcommand{\scrP}{\mathscr{P}}
\newcommand{\scrQ}{\mathscr{Q}}
\newcommand{\scrR}{\mathscr{R}}
\newcommand{\scrS}{\mathscr{S}}
\newcommand{\scrT}{\mathscr{T}}
\newcommand{\scrU}{\mathscr{U}}
\newcommand{\scrV}{\mathscr{V}}
\newcommand{\scrW}{\mathscr{W}}
\newcommand{\scrX}{\mathscr{X}}
\newcommand{\scrY}{\mathscr{Y}}
\newcommand{\scrZ}{\mathscr{Z}}

\newcommand{\calA}{\mathcal{A}}
\newcommand{\calB}{\mathcal{B}}
\newcommand{\calC}{\mathcal{C}}
\newcommand{\calD}{\mathcal{D}}
\newcommand{\calE}{\mathcal{E}}
\newcommand{\calF}{\mathcal{F}}
\newcommand{\calG}{\mathcal{G}}
\newcommand{\calH}{\mathcal{H}}
\newcommand{\calI}{\mathcal{I}}
\newcommand{\calJ}{\mathcal{J}}
\newcommand{\calK}{\mathcal{K}}
\newcommand{\calL}{\mathcal{L}}
\newcommand{\calM}{\mathcal{M}}
\newcommand{\calN}{\mathcal{N}}
\newcommand{\calO}{\mathcal{O}}
\newcommand{\calP}{\mathcal{P}}
\newcommand{\calQ}{\mathcal{Q}}
\newcommand{\calR}{\mathcal{R}}
\newcommand{\calS}{\mathcal{S}}
\newcommand{\calT}{\mathcal{T}}
\newcommand{\calU}{\mathcal{U}}
\newcommand{\calV}{\mathcal{V}}
\newcommand{\calW}{\mathcal{W}}
\newcommand{\calX}{\mathcal{X}}
\newcommand{\calY}{\mathcal{Y}}
\newcommand{\calZ}{\mathcal{Z}}

%CATEGORIES, SPECTRA
\newcommand{\DM}{\mathbf{DM}}
\newcommand{\KGL}{\mathsf{KGL}}
\newcommand{\SemiGrp}{\mathrm{SemiGrp}}
\newcommand{\Ab}{\mathrm{Ab}}
\newcommand{\CMon}{\mathrm{CMon}}
\newcommand{\Sch}{\mathrm{Sch}}
\newcommand{\VB}{\mathrm{VB}}
\newcommand{\modbar}{\mathrm{mod}\text{-}}
\newcommand{\barmod}{\text{-}\mathrm{mod}}
\newcommand{\baralg}{\text{-}\mathrm{alg}}
\newcommand{\Sm}{\mathrm{Sm}}
\newcommand{\SmOp}{\mathrm{SmOp}}
\newcommand{\calSH}{\mathcal{SH}}
\newcommand{\bfSH}{\mathbf{SH}}
\newcommand{\SH}{\mathbf{SH}}
\newcommand{\calMS}{\mathcal{MS}}
\newcommand{\bfMS}{\mathbf{MS}}
\newcommand{\MS}{\mathbf{MS}}
%\newcommand{\Ho}{\mathbf{Ho}}
\newcommand{\Ho}{\mathrm{Ho}}
\newcommand{\Spc}{\mathrm{Spc}}
\newcommand{\Spt}{\mathrm{Spt}}
\newcommand{\Pre}{\mathrm{Pre}}
\newcommand{\sPre}{s\mathrm{Pre}}
\newcommand{\Set}{\mathrm{Set}}
\newcommand{\sSet}{s\mathrm{Set}}
\newcommand{\Shv}{\mathrm{Shv}}
\newcommand{\sShv}{\mathrm{sShv}}
\newcommand{\Cat}{\mathrm{Cat}}
\newcommand{\Et}{\mathrm{Ét}}
\newcommand{\fEt}{f\mathrm{Ét}}
\newcommand{\SBil}{\mathrm{SBil}}
\newcommand{\pos}{\mathrm{pos}}
\newcommand{\Top}{\mathrm{Top}}
\newcommand{\Vect}{\mathrm{Vect}}
\newcommand{\tensor}{\otimes}
\newcommand{\Ch}{\mathrm{Ch}}
\newcommand{\ev}{\mathrm{ev}}
\newcommand{\eff}{\mathrm{eff}}
\newcommand{\gm}{\mathrm{gm}}
\newcommand{\Mgm}{M_{\mathrm{gm}}}
\newcommand{\Mgmt}{\widetilde{M_{\mathrm{gm}}}}
\newcommand{\Tr}{\mathrm{Tr}}
\newcommand{\tr}{\mathrm{tr}}
%\newcommand{\Cor}{\mathrm{Cor}}
\newcommand{\PST}{\mathrm{PST}}
\newcommand{\PSh}{\mathrm{PSh}}
\newcommand{\pst}{presheaf with transfers}
\newcommand{\psts}{presheaves with transfers}
\newcommand{\NisEx}{Nisnevich Excisive}
\newcommand{\Sus}{\mathrm{Sus}}
\newcommand{\sing}{\mathrm{sing}}
\newcommand{\HoA}{\Ho^{\A^1}}
\newcommand{\SingA}{\Sing^{\A^1}}
\newcommand{\ExA}{\mathrm{Ex}^{\A^1}}
\newcommand{\piA}{\pi^{\A^1}}
\newcommand{\Gm}{\G_m}
\newcommand{\KR}{KR}
\newcommand{\ext}{\mathrm{ext}}
%\newcommand{\MS}{\mathcal{MS}}
\newcommand{\etale}{\'etale}
\newcommand{\MGL}{\mathsf{MGL}}
\newcommand{\red}{\mathrm{red}}
\newcommand{\can}{\mathrm{can}}
\newcommand{\pair}{\mathrm{pr}}
\newcommand{\DMA}{{\mathbf{DM}}_A}

%\newcommand{\valv}{\mathrm{v}}
%\newcommand{\valw}{\mathrm{w}}

\newcommand{\strek}{\textthreequartersemdash}
\DeclareMathOperator{\diag}{diag}
\DeclareMathOperator{\resprod}{\mathrlap{\coprod}{\prod}}
\DeclareMathOperator{\Gal}{Gal}
\DeclareMathOperator{\Char}{char}
\DeclareMathOperator{\obj}{obj}
\DeclareMathOperator{\im}{im}
\DeclareMathOperator{\hofib}{hofib}
\DeclareMathOperator{\hocofib}{hocofib}
%\DeclareMathOperator{\tr}{tr}
\DeclareMathOperator{\Cl}{Cl}
\DeclareMathOperator{\GL}{GL}
\DeclareMathOperator{\Sym}{Sym}
\DeclareMathOperator{\Gr}{Gr}
\DeclareMathOperator{\SL}{SL}
\DeclareMathOperator{\Ann}{Ann}
\DeclareMathOperator{\Ext}{Ext}
\DeclareMathOperator{\Tor}{Tor}
\DeclareMathOperator{\Hom}{Hom}
\DeclareMathOperator{\Fun}{Fun}
%\DeclareMathOperator{\Ho}{Ho} %%% homotopy category
\DeclareMathOperator{\Der}{Der}
\DeclareMathOperator{\PDer}{PDer}
\DeclareMathOperator{\coker}{coker}
\DeclareMathOperator{\Spec}{Spec}
\DeclareMathOperator{\pr}{pr}
\DeclareMathOperator{\Proj}{Proj}
\DeclareMathOperator{\bfProj}{\mathbf{Proj}}
\DeclareMathOperator{\bfSpec}{\mathbf{Spec}}
\DeclareMathOperator{\Supp}{Supp}
\DeclareMathOperator{\supp}{supp}
\DeclareMathOperator{\id}{id}
\DeclareMathOperator{\Pl}{Pl}
\DeclareMathOperator{\St}{St}
\DeclareMathOperator{\Nm}{Nm}
\DeclareMathOperator{\Pic}{Pic}
\DeclareMathOperator{\rank}{rank}
\DeclareMathOperator{\Max}{Max}
\DeclareMathOperator{\ord}{ord}
\DeclareMathOperator{\ob}{ob}

%\DeclareMathOperator{\obj}{obj}
\DeclareMathOperator{\Bl}{Bl}
\DeclareMathOperator{\Div}{div}
\DeclareMathOperator{\codim}{codim}
\DeclareMathOperator{\equi}{equi}
\DeclareMathOperator{\BGL}{BGL}
\DeclareMathOperator{\cKGL}{KGL}
\DeclareMathOperator{\Map}{Map}
\DeclareMathOperator{\Sing}{Sing}
\DeclareMathOperator{\Cov}{Cov}
\DeclareMathOperator{\Cor}{Cor}
\DeclareMathOperator{\Kor}{Kor}
\DeclareMathOperator{\varlim}{\displaystyle{\lim_{\longleftarrow}}}
\DeclareMathOperator{\varcolim}{\displaystyle{\lim_{\longrightarrow}}}
\DeclareMathOperator{\colim}{colim}
\DeclareMathOperator{\holim}{holim}
\DeclareMathOperator{\hocolim}{hocolim}
\DeclareMathOperator{\cone}{cone}
\DeclareMathOperator{\cyl}{Cyl}
\DeclareMathOperator{\Mod}{Mod}
\DeclareMathOperator{\sk}{sk}
\DeclareMathOperator{\res}{res}
\DeclareMathOperator{\shom}{\mathbf{Hom}} %% simplicial homotopy object
\DeclareMathOperator{\Th}{Th}
\DeclareMathOperator{\tho}{th}
%\DeclareMathOperator{\red}{red}
\DeclareMathOperator{\cof}{cof}
\DeclareMathOperator{\topp}{top}
\DeclareMathOperator{\hp}{hp}
\DeclareMathOperator{\Ev}{Ev}
\DeclareMathOperator{\Fr}{Fr}
\DeclareMathOperator{\CH}{CH}
\DeclareMathOperator{\rk}{rk}
\DeclareMathOperator{\Bez}{Bez}

\DeclareMathOperator{\injj}{inj}
\DeclareMathOperator{\proj}{proj}

\newcommand*\cocolon{%
        \nobreak
        \mskip6mu plus1mu
        \mathpunct{}%
        \nonscript
        \mkern-\thinmuskip
        {:}%
        \mskip2mu
        \relax
}

%\newcommand{\SheafHom}{\mathscr{H}\kern-3pt \calligraphic{om}}
\newcommand{\SheafHom}{\mathscr{H}\kern-3pt om}
%\DeclareMathOperator{\SheafHom}{Shom}
\SelectTips{cm}{10}  


\begin{document}

\title{Cohomological correspondence categories}
\author{Andrei Druzhinin and Håkon Kolderup}
\maketitle


\section{Introduction}

\begin{definition}\label{def:coh}
Let $A^*$ be a twisted cohomology theory with supports on the category of smooth varieties over a noetherian separated base scheme $S$ of finite Krull dimension (check hypotheses). We say that $A^*$ is a \emph{good} cohomology theory if $A^*$ satisfies the following properties:
\begin{enumerate}
\item The cohomology theory is a \emph{ring cohomology theory}, i.e., there is an associative product structure
\[
A^n(X,\scrL)\otimes A^m(Y,\scrM)\to A^{n+m}(X\times_S Y,\scrL\boxtimes \scrM)
\]
and a unit $1\in A^0(S)$. (Remark: Equivalent to having cup product)
\item (Homotopy invariance) Let $X$ be a smooth $S$ scheme, let $\scrL$ be a line bundle on $X$ and let $Z$ be a closed subset of $X$. Then there is a natural isomorphism
\[
A^*_{Z\times\A^1}(X\times\A^1,\scrL)\cong A^*_Z(X,\scrL),
\]
where $p\colon X\times\A^1\to X$ is the projection.
\item (Pushforward) Suppose that $f\colon X\to Y$ is a morphism of smooth equidimensional $S$-schemes of dimensions $d_X$ respectively $d_Y$. Suppose moreover that $Z\subseteq X$ is a closed subset such that $f|_Z$ is finite. Then, for any $n\ge0$ and any line bundle $\scrL$ on $Y$, there is a pushforward homomorphism
\[
f_*\colon A^n_Z(X,\omega_f\otimes f^*\scrL)\to A^{n+d_Y-d_X}_{f(Z)}(Y,\scrL).
\]
\item (Excision) Suppose that $f\colon X\to Y$ is a flat morphism of smooth $S$-schemes. Assume moreover that $Z\subseteq Y$ is a closed subset such that $f|_{f^{-1}(Z)}\colon f^{-1}(Z)\to Z$ is an isomorphism. Then the pullback homomorphism
\[
f^*\colon A^n_Z(Y,\scrL)\to A^n_{f^{-1}(Z)}(X,f^*\scrL)
\]
is an isomorphism for any line bundle $\scrL$ on $Y$ and any $n\ge0$.
\item (Base change) Let
\[\begin{tikzcd}
X\ar{r}{\beta}\ar{d}{g} & X\ar{d}{f} \\
Y'\ar{r}{\alpha} & Y
\end{tikzcd}\]
be a Cartesian square of smooth $S$-schemes\todo{assumptions on the diagram? e.g., $f$ proper, $f$ and $\alpha$ Tor-independent?}. Then $g_*\alpha^*=\beta^*f_*$.
\item (Gersten resolution?)

\end{enumerate}
\end{definition}

To show:
\begin{itemize}
\item Given $\DMA(k)$, we can define $A$-motivic cohomology $H_A^{p,q}(X,\mathbb{Z})\defeq[M_A(X),\mathbb{Z}_A(q)[p]]_{\DMA(k)}$. Then $H_A^{p,p}(\Spec k,\mathbb{Z})\cong A^p(\Spec k)$. Maybe we here need to assume that the cohomology theory $A^*$ is defined as sheaf cohomology of a strictly $\A^1$-invariant Nisnevich sheaf (which, by Morel, has admits a Gersten resolution).
\item There is an adjunction $\gamma^*: \SH(k)\rightleftarrows \DMA(k):\gamma_*$, and the spectrum $\gamma_*(\mathbb{Z}_A)$ is effective.
\item Let $\mathscr{A}$ be a strict category of V-correspondences in the sense of Garkusha, and let $\DM_{\mathscr{A}}(k)$ be the associated derived category of motives, with Lefschetz motive $\mathbb{Z}_{\mathscr{A}}(1)$. Let $A^*$ be the cohomology theory defined as $A^n(X)=H^n_{\Nis}(X,\mathbb{Z}_{\mathscr{A}}(n))$. Then $A^*$ is a good cohomology theory, and there is an equivalence $\DMA(k)[1/e]\simeq\DM_{\mathscr{A}}(k)[1/e]$ after inverting the exponential characteristic.
\end{itemize}

\section{Cohomological correspondences}

Let $k$ be a field, and suppose that $A^*$ is a good cohomology theory. For $X,Y\in\Sm_k$, we write $X\times Y\defeq X\times_k Y$. Moreover, we abbreviate $\omega_{X\times Y/X}$ to $\omega_Y$.

\begin{definition}
Let $X$ and $Y$ be smooth connected $k$-schemes of dimension $d_X$ respectively $d_Y$. We define the group of \emph{$A$-correspondences from $X$ to $Y$} as
\[
\ACor_k(X,Y)\defeq\underset{T\in\calA(X,Y)}{\colim}A_T^{d_Y}(X\times Y,\omega_{Y}).
\]
If $X$ or $Y$ are not equidimensional, we sum over the equidimensional components.
\end{definition}

We will need to consider also a relative version of the above construction.

\begin{definition}
Suppose that $p\colon X\to S$ is a smooth map. Denote by $\calA_0(X/S)$ the set of \emph{admissible subsets of $X$ relative to $S$}---that is, closed subsets $T\subseteq X$ of $X$ such that each irreducible component of $T_\red$ is finite and surjective over $S$ via the map $p$. The set $\calA_0(X/S)$ is partially ordered by inclusions. As the empty set has no irreducible components, it is admissible.
\end{definition}

If $p\colon X\to S$ is smooth of relative dimension $n$, we define a covariant functor of abelian groups
\[
\calA_0(X/S)\to\Ab
\]
by $T\mapsto A_T(X,\omega_{X/S})$. If $T'\subseteq T$, we have an extension of supports homomorphism $A_{T'}(X,\omega_{X/S})\to A_{T}(X,\omega_{X/S})$. We will consider the colimit of this functor.

\begin{definition}\label{def:rel}
Suppose that $p\colon X\to S$ is a smooth map of relative dimension $n$, with $S$ equidimensional. Assume first that $X$ is also equidimensional. We let
\[
C^A_0(X/S)\defeq \underset{T\in\calA_0(X/S)}{\colim} A_T^{n}(X,\omega_{X/S})
\]
denote the group of \emph{relative $A$-correspondences}.

If $X$ is not equidimensional, we may write $X=\coprod_j X_j$ where the $X_j$'s are the equidimensional components of $X$. We then set
\[
C^A_0(X/S)\defeq\prod_j C^A_0(X_j/S).
\]
\end{definition}

The groups $C_0^A(X/S)$ allow us to define a category $\ACor_S$ of relative $A$-correspondences, in which the composition is constructed in a similar fashion as \cite{Calmes-Fasel}:

\begin{definition}
For $S$ a base scheme\todo{do we need that S is smooth over a field?}, let $\ACor_S$ denote the category whose objects are the same as the objects of $\Sm_S$, i.e., smooth separated schemes of finite type over $S$, and morphisms defined as follows. Let $X,Y\in\Sm_S$, and suppose first that $X$ and $Y$ are equidimensional. We then let
\[
\ACor_S(X,Y)\defeq C^A_0(X\times_S Y/X).\]
If $X$ or $Y$ are not equidimensional, let $X=\coprod_iX_i$ and $Y=\coprod_jY_j$ be the equidimensional decomposition of $X$ and $Y$. Then we put $\ACor_S(X,Y)\defeq\prod_{i,j}\ACor_S(X_i,Y_j)$.

Furthermore, defining $X\oplus Y\defeq X\amalg Y$ for any $X,Y\in\Sm_S$ turns $\ACor_S$ into an additive category with the empty scheme as zero-object. We refer to this category as the category of \emph{relative $A$-correspondences}.
\end{definition}

We have an embedding $\Sm_S\to \ACor_S$ defined identically as in \cite{Calmes-Fasel}. Furthermore, for $S$ a smooth $k$-scheme there is a functor $\ACor_k\to \ACor_S$ defined as follows. For any $X\in\Sm_k$, let $X_S\defeq X\times_k S$. Let $X,Y\in\Sm_k$; by working with one connected component at a time, we may assume that $X$ and $Y$ are connected. By the universal property of fiber products we have a morphism $f\colon X_S\times_S Y_S\to X\times Y$, which induces a pullback morphism
\[
f^*\colon A_T^{\dim Y}(X\times Y,\omega_{Y})\to A_{f^{-1}(T)}^{\dim Y}(X_S\times_S Y_S,f^*\omega_{Y})
\]for any $T\in\calA_0(X\times_k Y/X)=\calA(X,Y)$. As finiteness and surjectivity are preserved under base change we have $f^{-1}(T)\in\calA_0(X_S\times_S Y_S/X_S)$. Now, recall that $X$ and $Y$ are assumed to be smooth of relative dimension $\dim X$ respectively $\dim Y$ over $\Spec k$. By base change for smooth morphisms, $X_S\to S$ is smooth of relative dimension $\dim X$, and similarly $Y_S\to S$ is smooth of relative dimension $\dim Y$. It then follows from \cite[III Proposition 10.1]{Hartshorne} that $X_S\times_S Y_S$ is smooth over $X_S$ of relative dimension $\dim Y$.

Moreover, by smoothness, the canonical sheaf $\omega_{X/k}$ pulls back over $X_S$ to $\omega_{X_S/S}$ and similarly for $\omega_{Y/k}$. Hence $f^*\omega_{X\times Y/X}\cong\omega_{X_S\times_S Y_S/X_S}$. Since pullbacks commute with extension of support, we get an induced map on the colimit
\[
\ACor_k(X,Y)\to C^A_0(X_S\times_S Y_S/X_S)=\ACor_S(X_S,Y_S).
\]
All in all, we obtain a functor $\ACor_k\to \ACor_S$.


In the opposite direction there is a ``forgetful'' functor $\ACor_S\to \ACor_k$ induced by pushforwards.
Indeed, let $X,Y\in \Sm_S$. 
%and consider pushforward homomorphism $A^{*}(X\times_S Y) \to A^{*+\dim S}()
Then there is a pullback diagram
$$\xymatrix{
X\times_S Y \ar[r]^{i_{X,Y}}\ar[d] & X\times_k Y\ar[d]\\
\Delta_S \ar[r]^i & S\times_k S
,}$$
where $\Delta_S\subset S\times S$ denotes diagonal. Moreover, we have isomorphisms
\[\omega(X\times_S Y)\otimes i^*(\omega(X\times Y))^{-1}= \omega(i_{X,Y}) \simeq \omega(i) = \omega(S)^{-1}.\]
Thus there is, for any $T\in\calA_0(X\times_SY/X)$, a pushforward homomorphism 
\begin{align*}
(i_{X,Y})_*\colon A^{\dim_S Y}_{T}(X\times_S Y,\omega_S (Y))=A^{\dim_S Y}_T(X\times_S Y,\omega(Y)\otimes \omega(S)^{-1})\to 
A^{\dim Y}_{i_{X,Y}(T)}(X\times Y, \omega(Y) ).
\end{align*}
Passing to the colimit, we thus obtain a map $\ACor_S(X,Y)\to\ACor_k(X,Y)$.


\begin{construction}
Suppose that there is a diagram
\[\begin{tikzcd}
\calC\ar{r}{f}\ar{d}[swap]{p} & \A^1\\
U &
\end{tikzcd}
\]
satisfying the following properties:
\begin{enumerate}
\item $p\colon \mathcal C\to U$ is a smooth morphism with fibres being of dimension one,
\item $f\colon \mathcal C\to \mathbb A^1$ is a regular map such that $F\defeq(f,p)\colon \mathcal C\to \mathbb A^1\times U$ is quasi-finite, 
\item $Z(f)=Z\amalg Z^\prime$ is finite over $U$, and
\item there is an isomorpism $\mu\colon \omega_{\mathcal C}\otimes\omega_U^{-1}=\mathcal O(\mathcal C)$
\end{enumerate} 
We can then define a finite $A$-correspondence $\langle f \rangle_Z\in ACor(U,\mathcal C)$ as follows:


%Consider the graph $\Gamma\subset \mathcal C\times\mathbb A^1$ of the map $F$ over $U$ 
%and 
Let $\Gamma_F$ denote the graph the map $F$ over $U$, with embedding $i\colon \Gamma_F\hookrightarrow  \mathcal C\times\mathbb A^1$. \todo{could we as well consider the graph of $f$ instead of the graph of $F$?}
Consider the pushforward homomorphism 
$A^0(\Gamma,\mathcal O(\Gamma)\otimes\omega_{\mathbb A^1}^{-1})\to A_{\Gamma}^{1}(\mathcal C\times\mathbb A^1,\mathcal O(\mathcal C\times\mathbb A^1))$. Using the trivializations $\omega_{\mathbb A^1}\simeq \mathcal O(\mathbb A)$ and $\mu$ we get a homomorphism 
$A^0(\Gamma,\mathcal O(\Gamma)\otimes\omega_{\mathbb A^1}^{-1})\to A_{\Gamma}^{1}(\mathcal C\times \mathbb A^1,\omega_{\mathcal C}\times \omega_U^{-1})$. 
Let $i_*(1)\in A_{\Gamma}^{1}(\mathcal C\times\mathbb A^1,\omega \mathcal C\times \omega U^{-1})$ denote the image of $1\in  A^0(\Gamma,\mathcal O(\Gamma)$.

Next, consider the base change along the zero section $i_0\colon U\times 0\to U\times\mathbb A^1$, and the pullback homomorphism \todo{should $i_0$ go from $\calC\times0$ instead of $U\times0$? So that $i_0^*$ goes from $A^1_\Gamma(\calC\times\A^1,...)$?}
$i_0^*\colon A_{\Gamma}^{1}(\mathcal C,\omega_{\mathcal C}\times \omega_U^{-1})\to A_{Z(f)}^{1}(\mathcal C,\omega_{ \mathcal C}\times \omega_U^{-1})$. Since $Z(f)=Z\amalg Z^\prime$ we have $A_{Z(f)}^{1}(\mathcal C,\omega_{ \mathcal C}\times \omega_U^{-1})= A_{Z}^{1}(\mathcal C,\omega_{ \mathcal C}\times \omega_U^{-1})\oplus A_{Z^\prime}^{1}(\mathcal C,\omega_{ \mathcal C}\times \omega_U^{-1})$. So define $\langle f\rangle_Z$ as the image of $i_*(1)$ under the composition homomorphism
$$A_{\Gamma}^{1}(\mathcal C,\omega_{\mathcal C}\times \omega_U^{-1})\to 
A_{Z(f)}^{1}(\mathcal C,\omega_{ \mathcal C}\times \omega_U^{-1})\to
A_{Z}^{1}(\mathcal C,\omega_{ \mathcal C}\times \omega_U^{-1})\to
ACor_U(U,\mathcal C)\to ACor(U,\mathcal C).$$
\end{construction} 
 

\section{Injectivity for local schemes}\label{section:inj-loc-sch}
The goal of this section is to prove the following theorem.

\begin{theorem}\label{thm:inj-loc-sch}
Let $X$ be a smooth $k$-scheme and $x\in X$ a closed point. Let $U\defeq\Spec\calO_{X,x}$ and write $\can\colon U\rightarrow X$ for the canonical inclusion. Let $i\colon Z\to X$ be a closed subscheme with $x\in Z$ and let $j\colon X\setminus Z\to X$ be the open complement. Then there exists a finite $A$-correspondence $\Phi\in\ACor_k(U,X\setminus Z)$ such that the diagram
\[\begin{tikzcd}
& X\setminus Z\ar[d,"j"]\\
U\ar[ur,"\Phi"]\ar[r,"\can"] & X
\end{tikzcd}\]
commutes up to homotopy.
\end{theorem}

Let $X^\circ\subseteq X$ be an open neighborhood of the point $x$, and let $Z^\circ\defeq Z\cap X^\circ$. Then it is enough to solve the problem for the triple $U$, $X^\circ$ and $X^\circ\setminus Z^\circ$; in particular, we may assume that $X$ is irreducible and that the canonical sheaf $\omega_{X/k}$ is trivial. In fact, we will shrink $X$ so that we are in the situation of a relative curve over a quasi-projective scheme:

\begin{theorem}[\protect{\cite[Proposition 1]{PSV}}]
Let $X$, $Z$ and $x\in Z$ be as in \Cref{thm:inj-loc-sch}. Then there is a Zariski open neighborhood $X^\circ\subseteq X$ of the point $x$, an open subscheme $B$ of $\PP^{\dim X-1}$ and a commutative diagram
\[\begin{tikzcd}
  X^\circ\ar[r]\ar[dr,swap,"p"] & \ol X^\circ\ar[d,"\ol p"] & X_\infty^\circ\ar[l,swap]\ar[dl,"p_\infty"]\\
& B &
\end{tikzcd}\]
satisfying the following properties:
\begin{enumerate}
\item[$(1)$] $\ol p$ is a smooth projective morphism, whose fibers are irreducible projective curves.
\item[$(2)$] $X^\circ_\infty\defeq\ol X^\circ\setminus X^\circ$, and $p_\infty\colon X^\circ_\infty\to B$ is finite étale.
\item[$(3)$] $p|_Z$ is finite and $Z\cap X_\infty=\varnothing$. 
\end{enumerate}
The morphism $p\colon X^\circ\to B$ is called an \em{almost elementary fibration}.
\end{theorem}

Following \cite[§7]{hty-inv}, we may shrink $X$ such that there exists an almost elementary fibration $p\colon X\to B$ and such that $\omega_{X/k}$ and $\omega_{B/k}$ are trivial, i.e., $\omega_{X/k}\cong\calO_X$ and $\omega_{B/k}\cong\calO_B$.
Let $\scrX\defeq X\times_BU$ and $\scrZ\defeq Z\times_BU$. Let also $p_X\colon\scrX\to X$ and $p_U\colon\scrX\to U$ be the projections onto $X$ and $U$, respectively, and let $d_X$ denote the dimension of $X$. Finally, let $\Delta$ denote the morphism $\Delta\defeq(\can,\id)\colon U\to \scrX$.

\begin{lemma}[\protect{\cite[Lemma 7.1]{hty-inv}}]\label{lemma:nice-map}
There exists a finite surjective morphism 
\[
H_\theta=(h_\theta,p_U)\colon \scrX\to \A^1\times U
\]
over $U$, such that if we let $\scrD_1\defeq H_\theta^{-1}(1\times U)$ and $\scrD_0\defeq H_\theta^{-1}(0\times U)$ denote the scheme-theoretic preimages, then the following hold:
\begin{enumerate}
\item[$(1)$] $\scrD_1\subseteq \scrX\setminus\scrZ$.
\item[$(2)$] $\scrD_0=\Delta(U)\amalg\scrD_0'$ with $\scrD_0'\subseteq\scrX\setminus\scrZ$.
\end{enumerate}
\end{lemma} 

We will use \Cref{lemma:nice-map} to produce the desired $A$-correspondence $\Phi$. The aim is to define $\Phi$ as the image $(H_\theta\times1)_*(p_X)$ of the projection $p_X\in A^{d_X}_{\Gamma_{p_X}}(\scrX\times X,\omega_X)$ under the pushforward map
\[
(H_\theta\times1)_*\colon A^{d_X}_{\Gamma_{p_X}}(\scrX\times X,\omega_{H_\theta\times1}\otimes\omega_X)\to A^{d_X}_{(H_\theta\times1)(\Gamma_{p_X})}(\A^1\times U\times X,\omega_X).
\]
To this end, we need a trivialization of $\omega_{H_\theta\times1}=\omega_{\scrX\times X/k}\otimes(H_\theta\times1)^*\omega_{\A^\times U\times X/k}^\vee$. Now, as $U$ is local we have $\omega_{U/k}\cong\calO_U$. Keeping in mind the discussion preceding \Cref{lemma:nice-map}, it follows that the relative bundle $\omega_{H_\theta\times1}$ is also trivial.   
Thus we may choose an isomorphism $\chi\colon\omega_{H_\theta\times1}\cong\calO_X$.

\begin{definition}
Let $p_X\in\ACor_k(\scrX,X)$ denote the projection. Using the trivialization $\chi$ above, we let $\scrH_\theta\in\ACor_k(\A^1\times U,X)$ denote the image of $p_X\in\ACor_k(\scrX,X)$ under the composition
\begin{align*}
 A^{d_X}_{\Gamma_{p_X}}(\scrX\times X,\omega_X)&\cong A^{d_X}_{\Gamma_{p_X}}(\scrX\times X,\omega_{H_\theta\times1}\otimes\omega_X)\\
&\xrightarrow{(H_\theta\times1)_*} A_{(H_\theta\times1)(\Gamma_{p_X})}^{d_X}(\A^1\times U\times X,\omega_X).
\end{align*}
\end{definition}

\begin{lemma}\label{lemma:support}
The morphism $H_\theta\times1$ maps $\Gamma_{p_X}\cong\scrX$ isomorphically onto its image. Let $\scrH_0\defeq\scrH_\theta\circ i_0$ and $\scrH_1\defeq\scrH_\theta\circ i_1$. Identifying $\scrX$ with its image in $\A^1\times U\times X$, we then have $\supp\scrH_\theta=\scrX$, $\supp\scrH_0=\scrD_0$, and $\supp\scrH_1=\scrD_1$.
\end{lemma}

\begin{proof}
If $y=((x,u),x),y'=((x',u'),x')\in\Gamma_{p_X}$ is such that
\[
(H_\theta\times1)(y)=(h_\theta(x,u),u,x)=(H_\theta\times1)(y')=(h_\theta(x',u'),x',u'),
\]
it follows that $x=x'$ and $u=u'$, hence $y=y'$. Thus we can consider $\scrX$ as a subscheme of $\A^1\times U\times X$ by $(x,u)\mapsto(h_\theta(x,u),u,x)$. Now, the $A$-correspondence $p_X$ is supported on $\Gamma_{p_X}$, hence $\supp\scrH_\theta=(H_\theta\times1)(\Gamma_{p_X})\cong\scrX$. We turn to the restrictions $\scrH_0$ and $\scrH_1$ of the homotopy $\scrH_\theta$. We have \todo{should be true} $\scrH_\theta\circ i_\epsilon=(i_\epsilon\times1)^*(\scrH_\theta)$, where $\epsilon=0,1$. It follows that $\supp\scrH_\epsilon=(i_\epsilon\times1)^{-1}((H_\theta\times1)(\Gamma_{p_X})),$ and this closed subset is determined by those points $(x,u)\in\scrX$ satisfying $h_\theta(x,u)=\epsilon$. In other words, $\supp\scrH_\epsilon=\scrD_\epsilon$.
\end{proof}

\begin{lemma}\label{lemma:finding-Phi}
The finite $A$-correspondence $\scrH_\theta$ is a homotopy from $\scrH_0=\can+j\circ \Phi_0'$ to $\scrH_1=j\circ\Phi_1$, where $\Phi_0',\Phi_1\in\ACor_k(U,X\setminus Z)$.
\end{lemma}

\begin{proof}
By Lemmas \ref{lemma:nice-map} and \ref{lemma:support} we have $\supp\scrH_0=\Delta(U)\amalg\scrD_0'$, where $\scrD_0'\subseteq\scrX\setminus\scrZ$. Thus we may write $\scrH_0=\alpha+\beta$ where $\alpha\in\ACor_k(U,X)$ is supported on $\Delta(U)$ and $\beta\in\ACor_k(U,X)$ is supported on $\scrD_0'$. Since $\supp\beta=\scrD_0'\subseteq\scrX\setminus\scrZ$, there exists \todo{to show} a unique finite $A$-correspondence  $\Phi_0'\in\ACor_k(U,X\setminus Z)$ such that $j\circ\Phi_0'=\beta$. Hence $\scrH_0$ is of the form $\scrH_0=\alpha+j\circ\Phi_0'$ for $\Phi_0'\in\ACor_k(U,X\setminus Z)$. The same reasoning shows that, since $\supp\scrH_1= \scrD_1\subseteq\scrX\setminus\scrZ$, there is a unique $A$-correspondence $\Phi_1\in\ACor_k(U,X\setminus Z)$ such that $\scrH_1=j\circ\Phi_1$. 

It therefore only remains to understand the finite $A$-correspondence $\alpha\in A_{\Delta(U)}^{d_X}(U\times X,\omega_{X})$. Recall that, by definition, 
\[
\scrH_0=(i_0\times1)^*(H_\theta\times1)_*(\Gamma_{p_X})_*(\ip1).
\]
Let $i_{\Delta(U)}$ and $i_{\scrD_0}$ denote the respective inclusions $i_{\Delta(U)}\colon\Delta(U)\subseteq \scrX$ and $i_{\scrD_0}\colon\scrD_0\subseteq\scrX$. Base change applied to the pullback square
\[\begin{tikzcd}
(\Delta(U)\amalg\scrD_0')\times X\ar{r}{i_{\scrD_0}\times1}\ar{d}[swap]{H_\theta|_{\scrD_0}\times1} & \scrX\times X\ar{d}{H_\theta\times1}\\
U\times X\ar{r}{i_0\times1} & \A^1\times U\times X
\end{tikzcd}\]
reveals that $\alpha=(H_\theta|_{\Delta(U)}\times1)_*(i_{\Delta(U)}\times1)^*(\Gamma_{p_X})_*(\ip1)$. Using that $\Delta\colon U\to\scrX$ is an isomorphism onto its image and that $H_\theta|_{\Delta(U)}\colon\Delta(U)\to U$ is an isomorphism, we may write 
\[\alpha=(\Delta\times1)^*(\Gamma_{p_X})_*(\ip1).\] 
Next, consider the pullback diagram
\[\begin{tikzcd}
U\ar{rr}{\Delta}\ar{d}[swap]{\Gamma_\can} & & \scrX\ar{d}{\Gamma_{p_X}}\\
U\times X\ar{rr}{\Delta\times1} & &\scrX\times X.
\end{tikzcd}\]
Using base change once more, we obtain $\alpha=(\Gamma_\can)_*(\ip1)=\wt\gamma_\can$.
\end{proof}

\begin{remark}
The homotopy $\scrH_\theta$ does indeed depend on the choice of trivialization of $\omega_{H_\theta\times1}$. However, in the case of the finite $A$-correspondence $\alpha$ in the proof of \Cref{lemma:finding-Phi} above, recall that $\alpha=(H_{\theta}|_{\Delta(U)}\times1)_*(i_{\Delta(U)}\times1)^*(\Gamma_{p_X})_*(\ip1)$. Now, the relative bundle of $H_\theta|_{\Delta(U)}\colon\Delta(U)\to U$ is \emph{canonically} trivial, so the pushforward $(H_\theta|_{\Delta(U)}\times1)_*$ does not depend on the choice of trivialization. Any other choice of a trivialization of $\omega_{H_\theta\times1}$ yielding another homotopy $\scrH'_\theta$ would then satisfy $\scrH'_0=\can+j\circ\Phi_0''$ for some $\Phi_0''\in\ACor_k(U,X\setminus Z)$, which is sufficient for our purposes.
\end{remark}

\begin{proof}[Proof of \Cref{thm:inj-loc-sch}]
In the notation of \Cref{lemma:finding-Phi}, let $\Phi\defeq\Phi_1-\Phi_0'\in\ACor_k(U,X\setminus Z)$. As $\scrH_\theta$ provides a homotopy from $\can+j\circ\Phi_0'$ to $j\circ\Phi_1$, it follows that $\can\simeq j\circ(\Phi_1-\Phi_0')=j\circ\Phi$.
\end{proof}


\vspace{70pt}


Let $k$ be a field and let $\Sm_k$ denote the category of smooth varieties.

\begin{theorem}
Suppose $A$ is a (twisted??) cohomology theory $X\mapsto A^*(X,L)$ satisfying following axioms
\begin{itemize}
\item ring structure
\item push forward
\item base change
\item homotopy invariance
\item (Gersten conjecture)
\end{itemize}
Then for an infinite perfect field $k$ there is a category of motives $\DMA(k)$ with standard properties.
\end{theorem}

\begin{definition}
Definition $ACor$.
\end{definition}

\begin{definition}
Definition $ACor_S$.
\end{definition}


\begin{construction}\label{con:transposeCor}
$f X\to Y$ is finite; $f^t$
\end{construction}

\begin{construction}\label{con:CorRelCurve}
$f\colon \mathcal C\to \A^1_U$ is q.f., $Z(f)$ is finite;
$\langle f\rangle_Z$
\end{construction}


\begin{theorem}
excision on affine line
\end{theorem}
\begin{lemma}
left inverse in the category of pairs
\end{lemma}
\begin{proof}

\end{proof}
\begin{lemma}

\end{lemma}
\begin{proof}
$f$
\end{proof}






\printbibliography
\end{document}